%!TEX root = main.tex
\section{Examples}

\paragraph{Spiral}
The following code computes the coordinates $(x, y)$ of a spiral.
\begin{lstlisting}
def spiral(x: Real, y: Real, k: Int): (Real, Real) = {
  require(loopCounter(k) && x*x + y*y < 100 &&
    -10 <= x && x <= 10 && -10 <= y && y <= 10 &&
    -10 <= ~x && ~x <= 10 && -10 <= ~y && ~y <= 10)

  if (k < MAX) {
    val x1 = (9.9*x - y)/10
    val y1 = (9.9*y + x)/10
    spiral(x1,y1,k+1)
  } else {
    (x, y)
  }

} ensuring (_ match {
  case (a, b) => -10 <= a && a <= 10 && -10 <= b && b <= 10
})
\end{lstlisting}

Our tool automatically verifies that the loop precondition is an inductive
invariant. Note that we verify that the condition holds for both the real
as well as the actual computation.

It is also able to compute the difference between the real and the
actual computation. For example, after 10 loop iterations, it computes an
error of $3.27e-14$ (for both $x, y$ the error is identical). In comparison,
computing the error with through loop unrolling, we obtain an upper bound for
the error of $4.78e-14$, which comes close, however the computation takes
about 80 times as much time.


\paragraph{Mean}

The following code models an online computation of the mean and the variance.
We model the fact that at each loop iteration we obtain a new fresh value
as a function call.

\begin{lstlisting}
@model
  def nextItem: Real = {
    realValue
  } ensuring (res => -1200 <= res && res <= 1200)


  def varianceRec(n: Real, m: Real, s: Real): (Real, Real) = {
    require(-1200 <= m && m <= 1200 && 0 <= s && s <= 1440000 && 2 <= n && n <= 1000 &&
      integer(n))

    if (n <= 998) {
      val x = nextItem
      val m_new = ((n - 1) * m + x) / n
      val s_new = ((n - 2)/(n - 1)) * s + ((m_new - m) * (m_new - m))/n
      varianceRec(n + 1, m_new, s_new)
    } else {
      (m, s)
    }

  } ensuring (_ match {
    case (a, b) => -1200.00 <= a && a <= 1200.00 && 0 <= b && b <= 1440000.00
    })
\end{lstlisting}




\paragraph{Harmonic Oscillator}
\begin{lstlisting}
def eulerRec(x: Real, v: Real, i: Int): (Real, Real) = {
    require(loopCounter(i) && -10.0 <= x && x <= 10.0 &&
     -10.0 <= v && v <= 10.0 &&
      0.5*(v*v + 2.3*x*x) <= 115)

    if (i < 100) {
      eulerRec(x + 0.1 * v, v - 2.3 * 0.1 * x, i + 1)
    } else {
      (x, v)
    }
  } ensuring (_ match {
    case (a, b) => -10 <= a && a <= 10 && -10 <= b && b <= 10
  })
\end{lstlisting}

\begin{lstlisting}
def rk4Rec(x: Real, v: Real, i: Int): (Real, Real) = {
    require(loopCounter(i) && -10.0 <= x && x <= 10.0 &&
     -10.0 <= v && v <= 10.0 &&
      0.5*(v*v + 2.3*x*x) <= 115)

    if (i < 100) {

      val k: Real = 2.3
      val h: Real = 0.1

      val k1x = v
      val k1v = -k*x

      val k2x = v - k*h*x/2.0
      val k2v = -k*x - h*k*v/2.0

      val k3x = v - k*h*x/2.0 - h*h*k*v/4.0
      val k3v = -k*x - k*h*v/2.0 + k*k*h*h*x/4.0

      val k4x = v - k*h*x - k*h*h*v/2.0 + k*k*h*h*h*x/4.0
      val k4v = -k*x - k*h*v + k*k*h*h*x/2.0 + h*h*h*k*k*v/4.0

      val x1 = x + h*(k1x + 2.0*k2x + 2*k3x + k4x)/6.0
      val v1 = v + h*(k1v + 2.0*k2v + 2*k3v + k4v)/6.0

      rk4Rec(x1, v1, i + 1)

    } else {
      (x, v)
    }

  } ensuring (_ match {
    case (a, b) => -10 <= a && a <= 10 && -10 <= b && b <= 10
  })
\end{lstlisting}

\paragraph{Sine with Newton}
\begin{lstlisting}
def newton(x: Real, k: Int): Real = {
  require(loopCounter(k) && -1.0 < x && x < 1.0 && -1.0 < ~x && ~x < 1.0)

  if (k < 10) {
    newton(x - (x - (x°°3)/6.0 + (x°°5)/120.0 + (x°°7)/5040.0) /
      (1 - (x*x)/2.0 + (x°°4)/24.0 + (x°°6)/720.0), k + 1)
  } else {
    x
  }

  } ensuring(res => -1.0 < x && x < 1.0 && -1.0 < ~x && ~x < 1.0)
\end{lstlisting}

\begin{lstlisting}
def newtonRealUnstable(x: Real, k: Int): Real = {
  require(loopCounter(k) && -1.2 < x && x < 1.2)

  if (k < 10) {
    newtonRealUnstable(x - (x - (x°°3)/6.0 + (x°°5)/120.0 + (x°°7)/5040.0) /
      (1 - (x*x)/2.0 + (x°°4)/24.0 + (x°°6)/720.0), k + 1)
  } else {
    x
  }

} ensuring(res => -1.2 < x && x < 1.2)
\end{lstlisting}