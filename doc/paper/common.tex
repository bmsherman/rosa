\usepackage{amsmath}
\usepackage{amssymb}
\usepackage{graphicx}
\usepackage{listings}
\usepackage[usenames,dvipsnames,svgnames,table]{xcolor}

%\usepackage{subcaption}

% comment out if you want automatic paragraph indenting
\setlength{\parindent}{0pt}

\newcommand{\lv}{\lVert}
\newcommand{\rv}{\rVert}
\newcommand{\tl}{\tilde}
\newcommand{\R}{\mathbb{R}}
\newcommand{\eqn}[1]{
 \begin{align*} #1
 \end{align*}}

\newcommand{\eqnnumbered}[1]{
 \begin{align} #1
 \end{align}}

\newcommand*\colvec[3]{
    \begin{pmatrix}#1\\#2\\#3\end{pmatrix}
}

\newcommand*\norm[1]{
  \lVert #1 \rVert
}

\newcommand*\partialDer[2]{
  \frac{\partial #1}{\partial #2}
}


%% To import in the preambule
%\usepackage{listings}

\lstdefinelanguage{ML}{
  alsoletter={*},
  morekeywords={datatype, of, if, *},
  sensitive=true,
  morecomment=[s]{/*}{*/},
  morestring=[b]"
}

% "define" Scala
\lstdefinelanguage{scala}{
  alsoletter={@=>},
  morekeywords={abstract, case, catch, class, def, do, else, extends, false, final, finally, for, if, implicit, import, match, new, null, object,
override, package, private, protected, requires, return, sealed, super, this, throw, trait, try, true, type, val, var, while, with, yield, domain,
postcondition, precondition,invariant, constraint, assert, forAll,  _, return, @generator, ensure, require, ensuring,=>,Real,
certainly, possibly, certify, errorBound, assertBound, jacobian, derivative},%in,
  sensitive=true,
  morecomment=[l]{//},
  morecomment=[s]{/*}{*/},
  showstringspaces=false,
  columns=fullflexible,
  mathescape=true,
  numberstyle=\tiny,
  basicstyle=\codestyle,
  numbersep=5pt,
  stepnumber=2,
  numbers=left,                   % where to put the line-numbers
  morestring=[b]"
}

%\newcommand{\codestyle}{\small\sffamily}
\newcommand{\codestyle}{\small}

\newcommand{\union}{\mbox{\tt ++}}
\newcommand{\difference}{\mbox{\tt --}}
%\newcommand{\RA}{\mbox{\tt =>}}
\newcommand{\RA}{\Rightarrow}
\newcommand{\EQ}{\mbox{\tt ==}}

% Default settings for code listings
\lstset{
%  frame=tb,
  language=scala,
%  aboveskip=3mm,
%  belowskip=3mm,
%  lineskip=-0.1em,
%  numberstyle=\footnotesize,      % the size of the fonts that are used for the line-numbers
%  stepnumber=2,                   % the step between two line-numbers. If it's 1 each line will be numbered
%  numbersep=2pt
}

%\newenvironment{scala}
%        {\lstset{language=scala}\begin{lstlisting}}
%        {\end{lstlisting}}

\newcommand{\pseudostyle}{\sffamily}
% or try smallcaps (yak!)
% \newcommand{\pseudostyle}{\sc}

\lstdefinelanguage{pseudo}{
  alsoletter={@,=,>},
  morekeywords={abstract, case, catch, class, def, do, else, extends, false, final, finally, for, if, implicit, import, match, new, null, object,
override, package, private, protected, requires, return, sealed, super, this, throw, trait, try, true, type, val, var, while, yield, domain, %with
postcondition, invariant, constraint, assert, forAll,  _, return, @generator, ensure, require, ensuring,=>, %precondition
certainly, possibly, certify, errorBound, assertBound, jacobian, derivative, Input, Output},%in,
  sensitive=true,
  morecomment=[l]{//},
  morecomment=[s]{/*}{*/},
  showstringspaces=false,
  columns=fullflexible,
  mathescape=true,
  numberstyle=\tiny,
  basicstyle=\pseudostyle,
  numbers=left,                   % where to put the line-numbers
  morestring=[b]"
}
