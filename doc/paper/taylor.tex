%!TEX root = main.tex
\section{From Lipschitz to Taylor}

\begin{itemize}
\item Lipschitz is a special case of Taylor expansion for the case k = 0
\item we can use the case k = 1 to get an upper bound on the propagated error
which depends on the input values
\item we observe that the second order taylor remainder is small, due to the fact
that we take the square of the initial errors, which are already small (in many cases)
\item for straight-line code, we get thus error estimates that are, in the worst case,
apparently identical to the Lipschitz estimated ones (as compared with the same norm),
but we can obtain much smaller estimates for values where the intermediate ranges remain small
\item this could give us a method to summarize the error propagation of the function,
possibly with further over-approximating the second partial derivatives for conciseness
\end{itemize}