%!TEX root = work-in-progress.tex
\section{Inductive Specifications}

Consider the case of a generic loop, iteratively applying a function $f$.
The following three definitions are equivalent.
\begin{figure}[h!]
  \centering
  \lstset{numbers=none}
  \begin{subfigure}[b]{0.25\textwidth}
    \begin{lstlisting}
var x1 = ???
var x2 = ???
while(i < N) {
  (x1, x2) = f(x1, x2)
}
(x1, x2)
    \end{lstlisting}
    \caption{}
  \end{subfigure}%
  \begin{subfigure}[b]{0.32\textwidth}
    \begin{lstlisting}
    val x1, x2 = ???

    def iter(x1, x2, i) = {
      if (i < N) {
        val (y1, y2) = f(x1, x2)
        iter(y1, y2, i + 1)
      } else {
        (x1, x2)
      }
    }
    \end{lstlisting}
    \caption{}
  \end{subfigure}
  \begin{subfigure}[b]{0.32\textwidth}
    \begin{lstlisting}
    val x1, x2 = ???

    def iter(x1, x2, i) = {
      if (i < N) {
        val (y1, y2) = iter(x1, x2, i + 1)
        f(y1, y2)
      } else {
        (x1, x2)
      }
    }
    \end{lstlisting}
    \caption{}
  \end{subfigure}
\end{figure}

\newpage
\subsection{Specification language for loops}
We propose the following DSL for writing the specific type of loops given above.
\begin{lstlisting}
@loopbound(5)
def spiral(x: Real, y: Real): (Real, Real) = {
    require(x*x + y*y < 100 && -10 <= x && x <= 10 && -10 <= y && y <= 10)

    iterate(x, y) {
      x <== (9.9*x - y) / 10.0
      y <-- (9.9*y + x) / 10.0
    }

  } ensuring (_ match {
    case (a, b) => -10 <= a && a <= 10 && -10 <= b && b <= 10
  })
\end{lstlisting}

The loop body is given by the block passed to \lstinline{iterate},
and the loop counter is implicit, i.e. the verification is performed
with respect to any number of loop iterations.

The variables that are being updated in every iteration are given by
\lstinline{x <==} or \lstinline{x <--} (it's not yet decided which one is better).

For verification purposes, we can translate the above into a recursive function
in the form of $(a)$ or $(b)$.

In the case of the tail-recursive definition $(b)$, we need to prove three verification
conditions:
\begin{itemize}
\item $P \wedge \text{ body-if } \to P'$

\item $Q \to Q$

\item $P \to Q$
\end{itemize}

In the case of formulation $(c)$, we need to show
\begin{itemize}
\item $P \to P$

\item $Q \wedge \text{ body-if } Q'$

\item $P \to Q$
\end{itemize}

Effectively, both recursive definitions require the same verification effort.,
i.e. we need to find an inductive invariant $I$ such that $I \wedge body \to I'$,
and $P \to I$ and $I \to Q$.
This is in particular the case when $I = P = Q$.

In the \lstinline{spiral} example, the condition $x^2 + y^2 < 100$ is crutial for proving
the loop inductive, but the programmer may only provide it in the precondition.
In the absence of the user's specification being inductive by itself,
we can choose one of the following strategies for constructing $I$.
\begin{enumerate}
\item $I = P \wedge Q$, i.e. we strengthen the condition. It may happen that if verification
succeeds, we have shown inductiveness for a smaller input range than what the user indicated initially.

\item $I = P \vee Q$, we weaken the condition, taking into account all possible behaviours.
The question is that we may not be able to prove much (as in the example above).
\end{enumerate}

\subsection{Finite precision ranges}
The above spiral example only dealt with real bounds. Now suppose we want to show that the bounds
are satisfied also for a finite precision computation:
\begin{lstlisting}
def spiralBoth(x: Real, y: Real): (Real, Real) = {
    require(x*x + y*y < 100 && -10 <= x && x <= 10 && -10 <= y && y <= 10)

    iterate {
      x <== (9.9*x - y) / 10.0
      y <== (9.9*y + x) / 10.0
    }

  } ensuring (_ match {
    case (a, b) => -10 <= a && a <= 10 && -10 <= b && b <= 10 &&
      -10 <= ~a && ~a <= 10 && -10 <= ~b && ~b <= 10
  })
\end{lstlisting}

This example works out of box with the previous definition as follows.
The loop invariant we want to prove becomes under $I = P \wedge Q$:
\eqn{
  x*x + y*y < 100 &\wedge -10 <= x \wedge x <= 10 \wedge -10 <= y \wedge y <= 10 \wedge\\
      &-10 <= ~a \wedge ~a <= 10 \wedge -10 <= ~b \wedge ~b <= 10
}
which can be proven using our approximation procedure, which computes sound
bounds for $~x$ and $~y$, under the assumption that loop begin the ideal real
and actual finite variables differ by at most roundoff, and for the ideal counter-
part the condition \lstinline{x*x + y*y < 100} holds.
It is a little unfortunate that this is currently implicit.


\subsubsection{Code generation}
We can choose in principle to generate code in any version $(a), (b), (c)$,
adding an explicit loop counter.



\subsection{Where are the errors?}
In the above examples we took into account roundoff errors in each loop iteration,
but not the propagation across iterations. We also do not compute any bound on the final error.

Errors depend on the number of iterations, and if we add a term \lstinline{ x +/- a}
in the pre- or postcondition, it will necessarily be part of the loop invariant.
This condition, however, will be impossible to prove, expect in the rare cases where
absolute errors become smaller and smaller in each iteration.

As is explained in Section~\ref{theory-errors}, the two important values are $\sigma$,
the roundoff error of computing one loop body, and $K$ the Lipschitz constant.
It would thus make sense to verify and treat the error propagation separately from the
range bounds. (DLS for how to express this TBD).
