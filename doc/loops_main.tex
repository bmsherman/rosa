
\documentclass[10pt]{article}
\usepackage[top=1in, bottom=1in, left=1in, right=1in]{geometry} % see geometry.pdf on how to lay out the page. There's lots.
\geometry{a4paper}

\usepackage{amsmath}
\usepackage{amssymb}
\usepackage{graphicx}
\usepackage{listings}
\usepackage[usenames,dvipsnames,svgnames,table]{xcolor}

%\usepackage{subcaption}

% comment out if you want automatic paragraph indenting
\setlength{\parindent}{0pt}

\newcommand{\lv}{\lVert}
\newcommand{\rv}{\rVert}
\newcommand{\tl}{\tilde}
\newcommand{\eqn}[1]{
 \begin{align} #1
 \end{align}}

\newcommand*\colvec[3]{
    \begin{pmatrix}#1\\#2\\#3\end{pmatrix}
}

\newcommand*\norm[1]{
  \lVert #1 \rVert
}

\newcommand*\partialDer[2]{
  \frac{\partial #1}{\partial #2}
}


%% To import in the preambule
%\usepackage{listings}

\lstdefinelanguage{ML}{
  alsoletter={*},
  morekeywords={datatype, of, if, *},
  sensitive=true,
  morecomment=[s]{/*}{*/},
  morestring=[b]"
}

% "define" Scala
\lstdefinelanguage{scala}{
  alsoletter={@=>},
  morekeywords={abstract, case, catch, class, def, do, else, extends, false, final, finally, for, if, implicit, import, match, new, null, object,
override, package, private, protected, requires, return, sealed, super, this, throw, trait, try, true, type, val, var, while, with, yield, domain,
postcondition, precondition,invariant, constraint, assert, forAll,  _, return, @generator, ensure, require, ensuring,=>,Real,
certainly, possibly, certify, errorBound, assertBound, jacobian, derivative},%in,
  sensitive=true,
  morecomment=[l]{//},
  morecomment=[s]{/*}{*/},
  showstringspaces=false,
  columns=fullflexible,
  mathescape=true,
  numberstyle=\tiny,
  basicstyle=\codestyle,
  numbersep=5pt,
  stepnumber=2,
  numbers=left,                   % where to put the line-numbers
  morestring=[b]"
}

%\newcommand{\codestyle}{\small\sffamily}
\newcommand{\codestyle}{\small}

\newcommand{\union}{\mbox{\tt ++}}
\newcommand{\difference}{\mbox{\tt --}}
%\newcommand{\RA}{\mbox{\tt =>}}
\newcommand{\RA}{\Rightarrow}
\newcommand{\EQ}{\mbox{\tt ==}}

% Default settings for code listings
\lstset{
%  frame=tb,
  language=scala,
%  aboveskip=3mm,
%  belowskip=3mm,
%  lineskip=-0.1em,
%  numberstyle=\footnotesize,      % the size of the fonts that are used for the line-numbers
%  stepnumber=2,                   % the step between two line-numbers. If it's 1 each line will be numbered
%  numbersep=2pt
}

%\newenvironment{scala}
%        {\lstset{language=scala}\begin{lstlisting}}
%        {\end{lstlisting}}

\newcommand{\pseudostyle}{\sffamily}
% or try smallcaps (yak!)
% \newcommand{\pseudostyle}{\sc}

\lstdefinelanguage{pseudo}{
  alsoletter={@,=,>},
  morekeywords={abstract, case, catch, class, def, do, else, extends, false, final, finally, for, if, implicit, import, match, new, null, object,
override, package, private, protected, requires, return, sealed, super, this, throw, trait, try, true, type, val, var, while, yield, domain, %with
postcondition, invariant, constraint, assert, forAll,  _, return, @generator, ensure, require, ensuring,=>, %precondition
certainly, possibly, certify, errorBound, assertBound, jacobian, derivative, Input, Output},%in,
  sensitive=true,
  morecomment=[l]{//},
  morecomment=[s]{/*}{*/},
  showstringspaces=false,
  columns=fullflexible,
  mathescape=true,
  numberstyle=\tiny,
  basicstyle=\pseudostyle,
  numbers=left,                   % where to put the line-numbers
  morestring=[b]"
}


\setlength{\parindent}{0pt}


\newcommand{\openquestion}[1]{
 {\bf \colorbox{black}{{\color{orange}Open question}}} \fbox{#1}
  }

\title{Loops and Uncertainties}
\author{}

\begin{document}

\maketitle


\section{Proving bounds (without uncertainties)}
\begin{lstlisting}
def step(x: Real, y: Real, k: Real): (Real, Real) = {
  require(0 <= k && k <= 10 && -10 <= x && x <= 10 && -10 <= y && y <= 10 &&
            x*x + y*y < 100)

  // k should be an integer
  if (k > 1.1) {
    val x1 = (9*x - y)/10
    val y1 = (9*y + x)/10
    step(x1,y1,k-1)
  } else {
    (x, y)
  }

} ensuring (_ match {
  case (a, b) => -10 <= a && a <= 10 && -10 <= b && b <= 10
})
\end{lstlisting}

We can prove the bounds in the above example, by assuming that we start each iteration
``fresh'', i.e. we assume initial (roundoff) error is zero and show that the precondition
of the recursive call as well as the postcondition are satisfied, both for real and finite-precision
numbers.

{\bf Semantics of spec:} the written specification holds over the reals and we want to show the
same holds for floats, i.e. we replace all variables and operations by their floating-point ones.
Note that we so have two programs, side by side, without any connection/relation between the variables.

\section{Types of Loops}

\section{Uncertainties}
The overall error when computing the function $f$ is given by
$$
|f(a) - \tilde{f}(\tilde{a})|
$$
where $\tilde{}$ denotes the finite precision version.
From this we get
\begin{align}
&|f(a) - \tilde{f}(a) + \tilde{f}(a) - \tilde{f}(\tilde{a})|\\
\le& |f(a) - \tilde{f}(a)| + |\tilde{f}(a) - \tilde{f}(\tilde{a})|
\end{align}
The first difference is the roundoff error between the real function and
it's finite precision version, assuming no roundoff errors on the input.
We can already compute this difference with out approximation procedure.

The second difference is the propagation of existing errors. We'd like to
show that it is bounded by some function $g$:
$$
|\tilde{f}(a) - \tilde{f}(\tilde{a})| \le g(|a - \tilde{a}|)
$$
A special case of $g$ is when $\tilde{f}$ is Lipschitz continuous, when
$g(x) = L x$


{\bf Notation:}
\begin{itemize}
\item $\lambda$ denotes the initial error $|a - \tilde{a}| \le \lambda$
\item $\sigma$ denotes the roundoff error from one iteration, with exact inputs (no roundoff errors)
\item $k$ denotes the number of iterations
\item $\rho_f$ denotes the roundoff in computing $f$
\end{itemize}


\subsection{Dependence on number of iterations}
Iterating the function 3 times, i.e. $|f^3(a) - \tilde{f}^3(\tilde{a})|$
we get
\begin{align}
|f^3(a) - \tilde{f}^3(a) + \tilde{f}^3(a) - \tilde{f}^3(\tilde{a})| &\le
  \sigma + g(\sigma + g(\sigma)) + g^3(|a - \tilde{a}|) \\
&= \sigma + g(\sigma) + g^2(\sigma)+ g^3(\lambda)
\end{align}
assuming $g$ is a well-behaved function.
Note that we get the same result when using a different strategy:
$|f(f(a)) - \tilde{f}(f(a)) + \tilde{f}(f(a)) - \tilde{f}(\tilde{f}(\tilde{a}))|$

\subsubsection{Multivariate functions}
The above extends point-wise to multivariate functions $f: \mathbb{R}^n \to \mathbb{R}^n$
where initial errors are given component-wise $|x_i - \tilde{x}_i| \le \lambda_i$,
and roundoff errors for one loop iteration are also given component-wise $\sigma_i$.
Similarly, $g$ is now also vector-valued.


\openquestion{What happens as n goes to infinity? How does this compare to what we actually expect?}


\subsection{Showing a (finite precision) function Lipschitz continuous}
We want to show
$$
|\tilde{f}(a) - \tilde{f}(\tilde{a})| \le L |a - \tilde{a}|
$$
where $|a - \tilde{a}| \le \lambda$, where $\lambda$ is given (by the user or otherwise).

\openquestion{}
\begin{enumerate}
\item Using the Mean Value Theorem, and assuming the function is continuously differentiable, then
$$
L = sup |f'(\theta)| \quad \text{ where } \theta \in [a, b]
$$
\begin{itemize}
\item could work for floating-points and functions without branches, where we construct $\tilde{f}$ according to
our real to floating-point translation
\item However, $\tilde{f}$ is an interval valued function ($\delta_i$'s are bounded
by the machine epsilon).
\item Another option would be to compute $L$ (for the real-valued $f$) and from this somehow compute $\tilde{L}$
by adding roundoff errors.
\end{itemize}


\item Using an floating-point SMT solver, we want to show that
$
|\tilde{f}(x) - \tilde{f}(y)| > L |x - y|
$
is unsat. The FP solver in Z3 or Mathsat don't succeed in showing unsatisfiability for the case where
$f(x) = x*x$. Satisfiability seems to work (at least with Z3), but unsatisfiability is unreasonable.


\item Next, we could try our approximation as
$$
|f(x) \pm [\rho_{fx}] - (f(y) \mp [\rho_{fy}])| > L |x - y|
$$
Since $x, y$ are any numbers in the interval $[a, b]$, this gives us the approximation:
$$
|f(x) - f(y) \pm [2\rho_{f}])| > L |x - y|
$$
which is then a constraint over the reals.
\end{enumerate}

Note that the above will in principle work for any ``template function'' for g, not only for the Lipschitz
condition.
\openquestion{which template is reasonable?}


\subsubsection{Fixed-point and piece-wise defined functions}
Strategy 1 and 2 would only work for fixed-point arithmetic if we can use a bitvector procedure.
Scalability could become an issue...
The third strategy should work in all cases though.


\section{(Semantics of) Specification language}
A sample loop (x can be vector valued):
\begin{lstlisting}
  i = 0
  x = init
  while ( i < n ) {
    x = f(x)
  }
\end{lstlisting}


\subsection{Further considerations}
\begin{itemize}
\item a precondition $x == 4$ cannot be translated into $x == 4 \wedge x_{\circ} == 4$,
whereas for bounds this may make sense
\item How do parametric error specs compare with loop unrolling?
\end{itemize}


\openquestion{Special syntax for loops with a counter}\\

\openquestion{How to specify the error}\\
Possibilities are to only specify the one-loop bound and the Lipschitz condition,
but this won't say much. Or, we could specify the complete expression for $g$.
Or something in between

\subsection{Non-tailrecursive formulation}
Independent of the front-end syntax, we can transform the loops into a tail-recursive
formulation (as the first Ellipsoid example), or into a non-tailrecursive version.

\begin{lstlisting}
def R(x, i) {
  if (i >= n) x
  else {
    val xx = R(x, i + 1)
    f(xx)
  }
}
\end{lstlisting}
Note, that the precondition of the recursive call is now easily verified,
and we ``only'' need to verify the postcondition.



\section{Examples}
\begin{itemize}
\item harmonic oscillator (Euler, RK2, RK4)

\item nbody simulation (non-linear)

\item predator-prey (non-linear) {\color{red} TODO: Runge-Kutta}

\item {\color{red} TODO: dReal benchmarks?}

\item {\color{red} TODO: Keymaera?}
\end{itemize}


\openquestion{Can we formulate the update functions of PDE mesh-based method?}
Some of the inputs will only be known to be within some interval, but we may be
able to provide some estimate nonetheless. Some of the state will be carried over
from one iteration to the next.

\section{Extra features}

\openquestion{Adding method errors modeled as ``roundoff errors''}

\section{On a side note: conditionals}
Suppose we can show a function with branches to be Lipschitz continuous.
Then we have shown the error is bounded for all inputs, and we may use the same
distinction, but with piece-wise defined functions. (How to compute the Lipschitz
constant in this case is another matter.)

\end{document}
